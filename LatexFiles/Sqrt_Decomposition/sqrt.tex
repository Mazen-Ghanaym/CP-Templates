\documentclass[10pt]{article}
\usepackage{amsmath}
\usepackage{amsfonts}
\usepackage{amssymb}
\usepackage{geometry}
\usepackage{listings}
\usepackage{xcolor}
\usepackage[svgnames]{xcolor}
\geometry{a4paper, margin=1in}

% Code listing style
\lstset{
    language=C++,
    basicstyle=\ttfamily\small,
    keywordstyle=\color{blue},
    commentstyle=\color{DarkGreen},
    stringstyle=\color[rgb]{0.5,0,0.5},
    numbers=left,
    numberstyle=\tiny,
    frame=single,
    breaklines=true,
    showstringspaces=false,
    frameround=tttt
}

\title{Square Root Decomposition Ideas}
\author{Mazen Ghanayem}
\date{August 13, 2025}

\begin{document}

\maketitle

\section{Ideas}
\begin{itemize}
\item {
    Update Query: $k$ $v$, $a[k] \leftarrow v$ \\
    Range Frequency: $l$ $r$ $x$, count Frequency of $x$ in range $[l, r]$ \\
    
    \textbf{Implementation Details:}
    \begin{itemize}
    \item Preprocess the array into blocks of size $\sqrt{n}$.
    \item Maintain a frequency map for each block.
    \item For a query, combine results from relevant blocks.
    \end{itemize}
}
\rule{\linewidth}{1pt}
\item {
    Update Query: $k$ $v$, $a[k] \leftarrow v$ \\
    Query: $l$ $r$ $c$, count elements in range $[l, r]$ greater than or equal to $c$.

    \textbf{Implementation Details:}
    \begin{itemize}
    \item Preprocess the array into blocks of Ordered Multiset of size $\sqrt{n}$.
    \end{itemize}
}
\rule{\linewidth}{1pt}
\item {
    Update Query: \texttt{0 a b}, set power of hole $a$ to value $b$. \\
    Jump Query: \texttt{1 a}, find the last hole visited and the total number of jumps before the ball leaves the row.

    \textbf{Implementation Details:}
    \begin{itemize}
        \item Preprocess the array of $N$ holes into blocks of size $\sqrt{N}$.
        \item For each hole $i$, precompute a pair of values: (1) the next hole the ball lands in \textit{outside} its current block, and (2) the number of jumps it takes to get there.
        \item A jump query can then be answered in $O(\sqrt{N})$ time by chaining these precomputed block-to-block jumps.
        \item An update only requires recomputing the values for the single block that was changed, also in $O(\sqrt{N})$ time.
    \end{itemize}
    \begin{lstlisting}[language=C++, basicstyle=\ttfamily\fontsize{8pt}{7pt}\selectfont]
    void process() {
        for (int i = n - 1; i >= 0; i--) {
            int idx = i / SQ, r = min(n - 1, (i / SQ + 1) * SQ - 1LL);
            if (i + arr[i] > r) {
                jumps[i] = {i, 1};
            } else {
                jumps[i] = jumps[i + arr[i]];
                jumps[i].second++;
            }
        }
    }

    void update(int idx, ll val) {
        arr[idx] = val;
        int blk_idx = idx / SQ;
        int l = blk_idx * SQ, r = min(n - 1, (blk_idx + 1) * SQ - 1LL);
        for (int i = r; i >= l; i--) {
            if (i + arr[i] > r) {
                jumps[i] = {i, 1};
            } else {
                jumps[i] = jumps[i + arr[i]];
                jumps[i].second++;
            }
        }
    }

    pair<int, int> query(int l) {
        pair<int, int> ans = {0, 0};
        int idx = l;
        while (idx < n) {
            ans.first = jumps[idx].first;
            ans.second += jumps[idx].second;
            idx = jumps[idx].first + arr[jumps[idx].first];
        }
        return ans;
    }
    \end{lstlisting}
}
\rule{\linewidth}{1pt}
\item {
    Range Update: \texttt{L R X}, add $X$ to heights of hills in range $[L, R]$. \\
    Jump Query: \texttt{i k}, find the final hill after making $k$ jumps starting from hill $i$. A jump is to the nearest hill on the right that is strictly taller and within a distance of 100.

    \textbf{Implementation Details:}
    \begin{itemize}
        \item Divide the array into blocks of size $\sqrt{N}$. Use a lazy array (e.g., \texttt{lazy\_add}) for efficient range updates on full blocks.
        \item For each hill $i$, precompute: the next immediate jump (\texttt{next\_jump}), the final hill reached if jumping only within the block (\texttt{block\_exit\_node}), and the number of jumps to get there (\texttt{jumps\_to\_exit}).
        \item This preprocessing is done efficiently in $O(\sqrt{N})$ per block by iterating backwards with a stack to find the next greater element.
        \item A jump query uses the precomputed data to skip across blocks and performs a small linear scan for inter-block jumps, leading to an overall $O(\sqrt{N})$ query time.
    \end{itemize}
    \begin{lstlisting}[language=C++, basicstyle=\ttfamily\fontsize{7pt}{8pt}\selectfont]
    #include <bits/stdc++.h>
    using namespace std;
    using ll = long long;

    const int MAX_N = 100005, MAX_SQRT_N = 320, MAX_JUMP_DIST = 100;
    int n, q, block_size;
    ll heights[MAX_N], lazy_add[MAX_SQRT_N];
    int next_jump[MAX_N], block_exit_node[MAX_N], jumps_to_exit[MAX_N];

    void rebuild_block(int block_idx) {
        int start_idx = block_idx * block_size;
        int end_idx = min((block_idx + 1) * block_size, n);

        if (lazy_add[block_idx] != 0) {
            for (int i = start_idx; i < end_idx; ++i) heights[i] += lazy_add[block_idx];
            lazy_add[block_idx] = 0;
        }

        stack<int> s;
        for (int i = end_idx - 1; i >= start_idx; --i) {
            while (!s.empty() && (s.top() - i > MAX_JUMP_DIST || heights[s.top()] <= heights[i])) {
                s.pop();
            }

            if (!s.empty()) {
                next_jump[i] = s.top();
                block_exit_node[i] = block_exit_node[next_jump[i]];
                jumps_to_exit[i] = jumps_to_exit[next_jump[i]] + 1;
            } else {
                next_jump[i] = i;
                block_exit_node[i] = i;
                jumps_to_exit[i] = 0;
            }
            s.push(i);
        }
    }

    void update_range(int l, int r, ll val) {
        int start_block = l / block_size;
        int end_block = r / block_size;

        if (start_block == end_block) {
            for (int i = l; i <= r; ++i) heights[i] += val;
            rebuild_block(start_block);
            return;
        }

        for (int i = l; i < (start_block + 1) * block_size; ++i) heights[i] += val;
        rebuild_block(start_block);

        for (int i = start_block + 1; i < end_block; ++i) lazy_add[i] += val;

        for (int i = end_block * block_size; i <= r; ++i) heights[i] += val;
        rebuild_block(end_block);
    }

    int query_jumps(int start_idx, int k) {
        int current_hill = start_idx;

        while (k > 0) {
            // Case 1: We don't have enough jumps to take the full precomputed path
            // We must take a single, precomputed step instead.
            if (jumps_to_exit[current_hill] > 0 && k < jumps_to_exit[current_hill]) {
                current_hill = next_jump[current_hill];
                k--;
                continue;
            }

            // Case 2: We have enough jumps to use the precomputed path.
            if (jumps_to_exit[current_hill] > 0) {
                k -= jumps_to_exit[current_hill];
                current_hill = block_exit_node[current_hill];
            }

            // Case 3: We are now at a block's exit point and still have jumps left
            // We must perform one jump by scanning manually.
            if (k > 0) {
                bool jumped = false;
                ll current_true_height = heights[current_hill] + lazy_add[current_hill / block_size];

                for (int j = current_hill + 1; j < min(n, current_hill + 1 + MAX_JUMP_DIST); ++j) {
                    ll next_true_height = heights[j] + lazy_add[j / block_size];
                    if (next_true_height > current_true_height) {
                        current_hill = j;
                        k--;
                        jumped = true;
                        break;
                    }
                }
                if (!jumped) break; // No further jump possible, we are stuck.
            }
        }
        return current_hill + 1;
    }

    int main() {
        ios_base::sync_with_stdio(false), cin.tie(NULL);

        cin >> n >> q;
        block_size = static_cast<int>(sqrt(n));
        if (block_size == 0) block_size = 1;

        for (int i = 0; i < n; ++i) {
            cin >> heights[i];
        }

        for (int i = (n - 1) / block_size; i >= 0; --i) {
            rebuild_block(i);
        }

        while (q--) {
            int type;
            cin >> type;
            if (type == 1) {
                int i, k;
                cin >> i >> k;
                cout << query_jumps(i - 1, k) << " \n";
            } else {
                int l, r;
                ll x;
                cin >> l >> r >> x;
                update_range(l - 1, r - 1, x);
            }
        }
        return 0;
    }
    \end{lstlisting}
}
\end{itemize}

\end{document}